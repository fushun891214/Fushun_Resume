%-------------------------
% Resume in LateX
% Author : Sourabh Bajaj
% License : MIT
%------------------------

\documentclass[letterpaper,11pt]{article}

\usepackage{xeCJK}
\setCJKmainfont{Noto Sans CJK TC}
\usepackage{latexsym}
\usepackage[empty]{fullpage}
\usepackage{titlesec}
\usepackage{marvosym}
\usepackage[usenames,dvipsnames]{color}
\usepackage{verbatim}
\usepackage{enumitem}
\usepackage[hidelinks,pdfnewwindow=true]{hyperref}
\usepackage{fancyhdr}
\usepackage{tabularx}

\pagestyle{fancy}
\fancyhf{} % Clear all header and footer fields
\fancyfoot{}
\renewcommand{\headrulewidth}{0pt}
\renewcommand{\footrulewidth}{0pt}

% Adjust margins
\addtolength{\oddsidemargin}{-0.5in}
\addtolength{\evensidemargin}{-0.5in}
\addtolength{\textwidth}{1in}
\addtolength{\topmargin}{-.5in}
\addtolength{\textheight}{1.0in}

\urlstyle{same}

\raggedbottom
\raggedright
\setlength{\tabcolsep}{0in}

% Sections formatting
\titleformat{\section}{
  \vspace{-4pt}\scshape\raggedright\large
}{}{0em}{}[\color{black}\titlerule \vspace{-5pt}]

%-------------------------
% Custom commands
\newcommand{\resumeItem}[2]{
  \item\small{
    \textbf{#1}{: #2 \vspace{-2pt}}
  }
}

% Just in case someone needs a heading that does not need to be in a list
\newcommand{\resumeHeading}[4]{
    \begin{tabular*}{0.99\textwidth}[t]{l@{\extracolsep{\fill}}r}
      \textbf{#1} & #2 \\
      \textit{\small#3} & \textit{\small #4} \\
    \end{tabular*}\vspace{-5pt}
}

\newcommand{\resumeSubheading}[4]{
  \vspace{-1pt}\item
    \begin{tabular*}{0.97\textwidth}[t]{l@{\extracolsep{\fill}}r}
      \textbf{#1} & #2 \\
      \textit{\small#3} & \textit{\small #4} \\
    \end{tabular*}\vspace{-5pt}
}

\newcommand{\resumeSubSubheading}[2]{
    \begin{tabular*}{0.97\textwidth}{l@{\extracolsep{\fill}}r}
      \textit{\small#1} & \textit{\small #2} \\
    \end{tabular*}\vspace{-5pt}
}

\newcommand{\resumeSubItem}[2]{\resumeItem{#1}{#2}\vspace{-4pt}}

\renewcommand{\labelitemii}{$\circ$}

\newcommand{\resumeSubHeadingListStart}{\begin{itemize}[leftmargin=*]}
\newcommand{\resumeSubHeadingListEnd}{\end{itemize}}
\newcommand{\resumeItemListStart}{\begin{itemize}}
\newcommand{\resumeItemListEnd}{\end{itemize}\vspace{-5pt}}

%-------------------------------------------
%%%%%%  CV STARTS HERE  %%%%%%%%%%%%%%%%%%%%%%%%%%%%


\begin{document}

%----------HEADING-----------------
\begin{tabular*}{\textwidth}{l@{\extracolsep{\fill}}r}
  \textbf{\href{https://web.fushun181.com/}{\Large 張富順}} & Email: \href{mailto:fushun891214@gmail.com}{fushun891214@gmail.com}\\
  \href{https://web.fushun181.com/}{web.fushun181.com} & 手機: \href{tel:+886968733583}{+886-968733583} \\
\end{tabular*}


%-----------EDUCATION-----------------
\section{學歷}
  \resumeSubHeadingListStart
    \resumeSubheading
      {國立臺北科技大學}{台北,台灣}
      {電子工程系碩士,計算機組(\href{https://drive.google.com/file/d/1fm6N8K4sa3hF5T6KT4cwoBWaxjz9KE2g/view?usp=sharing}{查看成績單})}{2024年7月 -- 2026年8月}
    \resumeSubheading
      {東吳大學}{台北,台灣}
      {資訊管理系學士(\href{https://drive.google.com/file/d/1x6KG0wfyhn3l8BxUbNuExm5QGhRpGS5u/view?usp=share_link}{查看成績單})}{2021年2月 -- 2024年3月}
  \resumeSubHeadingListEnd


%-----------CERTIFICATIONS-----------------
\section{證照}
  \resumeSubHeadingListStart
    \resumeSubItem{多益 TOEIC}
      {560 分(取得日期:2025年7月)(\href{https://drive.google.com/file/d/1cn7Se__A2mzFqkbpxQ4oiAXuJRGHWV6u/view?usp=sharing}{查看證書})}
  \resumeSubHeadingListEnd


%-----------EXPERIENCE-----------------
\section{實驗室經驗}
  \resumeSubHeadingListStart

    \resumeSubheading
      {MMSLAB}{台北,台灣}
      {實驗室成員}{2024年7月 -- 2026年8月}
      \resumeItemListStart
        \resumeItem{iTalkuTalk}
          {[\href{https://www.italkutalk.com/}{網站} | \href{https://apps.apple.com/tw/app/italkutalk/id1263409577}{App Store} | \href{https://play.google.com/store/apps/details?id=lab.italkutalk&hl=zh_TW}{Google Play}] iTalkuTalk 是實驗室開發維護的大型語言學習平台,在 Google Play 商店擁有超過100萬次的下載數量。 我在此專案中負責後端API開發、資料庫結構設計及網頁開發,接手並重構前人撰寫的程式碼,並持續依照使用者需求調整與優化功能。 碩一期間,我重構了專案的後台管理系統,原後台採用 EJS 架構開發,但因缺乏交接文件而難以維護, 因此我建立了新專案,改用 Vue 框架重新開發頁面並串接API,大幅提升了後台的 UI 介面與整體效能。}
        \resumeItem{HomeEasy}
          {[\href{https://www.homeeasy.app/}{網站} | \href{https://apps.apple.com/us/app/家易-home-easy-裝潢施工比價平台/id6477271187}{App Store} | \href{https://play.google.com/store/apps/details?id=lab.homeeasy}{Google Play}] HomeEasy 是實驗室開發維護的裝潢施工比價平台。提供網頁、Android 及 iOS 多個平台服務。 我在碩二時接手此專案,負責後端API開發、資料庫結構設計及網頁開發,並修正了多個功能與流程上的Bug, 針對專案曝光率不足的問題 ,我研究了 Google Search Console 及 SEO 檢測網站的建議, 優化網頁專案的 SEO 相關設定,大幅提升了專案的整體曝光。}
      \resumeItemListEnd

  \resumeSubHeadingListEnd


%-----------PROJECTS-----------------
\section{專案經歷}
  \resumeSubHeadingListStart
    \resumeSubItem{虛擬檔案系統(Virtual File System)}
      {[\href{https://github.com/fushun891214/Advance_C_Final_Project}{GitHub}] 進階 C 語言實務的期末專案,主要目的是要透過 C 語言實作 Virtual File System,包含 vi、ls、mkdir、rm 等指令。自定義 INode / SuperBlock 資料結構管理 Virtual Disk 的資料分配,運用 Pointer、Pointer to Pointer、Bitmap、動態記憶體管理等技術。專案採用分層式模組化架構設計,將功能劃分為三個獨立模組:\\
      - 核心層(space.c/h):實作虛擬磁碟的底層管理機制,包含 SuperBlock、INode 和 Block 的分配與釋放,提供檔案系統的基礎設施\\
      - 命令層(commands.c/h):建立在核心層之上,實作使用者命令介面(如 ls, cd, mkdir, rm 等),處理路徑解析、檔案操作及加密儲存功能\\
      - 編輯器層(vi.c/h):提供類 vi 文字編輯器功能,包含插入、刪除、儲存等操作,展現應用層的複雜功能實作\\
      此架構具有高內聚、低耦合特性,各模組職責明確,便於維護與擴充。}
    \resumeSubItem{Pay Off Bar}
      {[\href{https://youtu.be/O5BIyIPJP1Q?si=yKiBYVI9u3izC_13}{Demo 影片} | \href{https://github.com/fushun891214/payOffBarServer}{GitHub}] 作業系統的期末專案,主要是為了解決外出吃飯時的記帳問題,能夠自動記帳,避免買單人與被買單人遺忘帳款。主要有以下幾種功能:\\
      - 群組管理:能夠建立群組管理帳款紀錄,顯示所有被買單人的欠款金額與付款狀態,方便買單人與被買單人共同管理分帳\\
      - 好友系統:透過好友ID搜尋功能,讓使用者之間的建立好友關係\\
      - 付款提醒:透過即時推播功能,讓買單人透過手機推播提醒被買單人付款\\
      - 付款狀態:買單人可在群組內更新被買單人的付款狀態,並透過推播通知被買單人已完成付款\\
      在付清Bar 專案中,我負責後端API開發與資料庫結構設計。在好友系統與群組管理功能中,運用了多個資料集之間的關聯設計,並在後端 API Server 整合 Firebase Cloud Messaging 服務,使前端呼叫 API 時能同步觸發 App 的即時推播通知功能。}
    \resumeSubItem{股票資料爬蟲(Stock Crawler)}
      {[\href{https://github.com/fushun891214/Stock_Crawler}{GitHub}] 是資料工程實務與應用的期末專案,主要目標是透過爬蟲技術爬取臺灣證券交易所的資料, 並將所有爬取到的資料儲存至 MongoDB 資料庫中, 後續透過查詢 MongoDB 取得特定股票的成交資訊、收盤價、均價等資料, 並使用 Python 的 Matplotlib 套件進行資料視覺化呈現。}
    \resumeSubItem{旅遊網站(Travel Website)}
      {[\href{https://github.com/fushun891214/Travel-Website}{GitHub}] 是敏捷開發與實務應用的期末專案,主要目標是透過 Scrum 方法開發旅遊網站, 根據使用者需求,透過密集討論與快速迭代的方式,規劃並設計整個專案流程 最終實作出支援 RWD 響應式顯示及電子郵件驗證功能的旅遊網站專案。}
    \resumeSubItem{專題-皮膚識別 APP(Skin Identification App)}
      {[\href{https://github.com/fushun891214/Skin_Identification_}{GitHub}] 是我的大學專題,主要目標是透過手機鏡頭拍攝人體皮膚,辨識出各種皮膚病徵, 提供了註冊登入功能,讓使用者可以記錄並追蹤自身的皮膚健康狀態, 在此專案中,我負責整合皮膚辨識模型及 APP 功能開發,包括實作 Google 第三方註冊/登入功能與使用者資料管理。}
  \resumeSubHeadingListEnd

%
%--------SKILLS------------
\section{技能}
  \resumeSubHeadingListStart
    \item
      \textbf{後端開發}
      \resumeItemListStart
        \item\small{熟悉 .NET 與 Express API 開發}
        \item\small{熟悉 MVC 專案架構}
        \item\small{DB 資料結構設計與維護}
        \item\small{Docker 建置部署專案能力}
        \item\small{版本控制工具,如 Git、Github 等}
        \item\small{CI/CD 自動化部署工具,如 Jenkins 等}
      \resumeItemListEnd
    \item
      \textbf{網頁開發}
      \resumeItemListStart
        \item\small{熟悉基本網頁結構:HTML、JS、CSS}
        \item\small{網頁框架:EJS、Vue、Nuxt}
        \item\small{熟悉網頁渲染模式:CSR 與 SSR}
        \item\small{網頁 SEO 最佳化設置,如 meta 元素設置}
        \item\small{API 串接,如透過 axios 進行 HTTP 請求}
        \item\small{設置個人化資料,並保存在 Local Storage}
      \resumeItemListEnd
    \item
      \textbf{雲服務開發}
      \resumeItemListStart
        \item\small{熟悉 AWS 服務操作,如 EC2、Load Balace、Lambda、ECR、S3 等}
        \item\small{熟悉 Cloudflare 服務操作,如 Domain Name 註冊與管理、Cloudflare Tunnel 配置、DNS 設置及 LoadBalance 串接等}
      \resumeItemListEnd
    \item
      \textbf{AI Agent 開發}
      \resumeItemListStart
        \item\small{熟悉建置 Ollama 環境運行 LLM}
        \item\small{熟悉透過 Langchain 開發 RAG}
        \item\small{使用 ChromaDB 儲存向量資料}
      \resumeItemListEnd
  \resumeSubHeadingListEnd


%-------------------------------------------
\end{document}
